\documentclass[a4paper,11pt]{article}
\usepackage[a4paper, total={6in, 10in}]{geometry}
\usepackage[utf8]{inputenc}
\usepackage{amsmath}
\usepackage{graphicx}
\usepackage{titlesec}
\usepackage{enumitem}
\usepackage[backend=biber]{biblatex}
\bibliography{referencias}
% Redefinir numeración de secciones y subsecciones
\renewcommand{\thesection}{\Roman{section}}
\renewcommand{\thesubsection}{\Roman{section}.\Roman{subsection}}
\renewcommand{\thesubsubsection}{\Roman{section}.\Roman{subsection}.\Roman{subsubsection}}

% Configuración de títulos de secciones
\titleformat{\section}[block]{\Large\bfseries\filcenter}{}{1em}{}

\title{
    \begin{center}
        \includegraphics[width=0.2\textwidth]{./logotipo-principal-rgb-color-2_904941-2048x492.png}\\ % Ajusta el tamaño y nombre del archivo según sea necesario
        \vspace{1cm}
        \large \textbf{INDIVIDUO Y MEDIO AMBIENTE (100000N09I)}\\
        \vspace{1cm}
        \large \textbf{Informe de Portafolio Final}\\
        \vspace{7cm}
        \Huge \textbf{Impacto del Turismo Informal en las Lomas Costeras de Lima y Estrategias de Mitigación Ambiental}
    \end{center}
}

\author{
    \textbf{Luis Huatay}\\
    \texttt{U24218809@utp.edu.pe}
}

\date{}

\begin{document}

\maketitle

\newpage

\section*{Índice}
\begin{enumerate}[label=\Roman*.]
    \item Descripción del ecosistema
    \item Impacto ambiental en el ecosistema
    \item Objetivos de desarrollo sostenible
    \item Propuesta de solución
    \item Conclusiones y recomendaciones
    \item Referencias
\end{enumerate}

\newpage

\section{I Descripción del ecosistema}
\subsection{Lomas costeras}
Ecosistema costero de desierto, conocido como “oasis de vegetación de neblinas”, 
que corresponde a formaciones vegetales xerófilas efímeras que incluyen herbáceas, 
con árboles dispersos en algunos casos y ricas en endemismos vegetales, que 
estacionalmente cubren extensas zonas desérticas en las colina y lomadas medianas 
expuestas a neblinas invernales, elevada humedad relativa por encima de 80\% y la 
captación de gotas de agua por la vegetación arbustiva y arbórea, desde los 100 m 
s. n. m. hasta cerca de 1000 m s. n. m., entre los 8° LS hasta los 18° LS (inmediaciones 
de Tacna).
\subsection{Lomas costeras de Lima}
Las lomas de Lima, se encuentran entre las más húmedas del Perú, debido a la presencia de neblinas durante el invierno. Estudios recientes del Instituto Metropolitano de Planificación (IMP) y el Servicio de Parques de Lima (SERPAR) de la 
Municipalidad Metropolitana de Lima registran 70,000 hectáreas de lomas, de las cuales aproximadamente 22,000 tienen una intensidad anual durante el invierno.
\section{II Impacto ambiental en el ecosistema}
\subsection{El impacto del turismo informal}
El turismo informal en las lomas costeras de Lima puede tener un impacto negativo en el medio ambiente debido a la falta de regulación y control de las actividades turísticas. Algunos de los impactos ambientales del turismo informal en las lomas costeras de Lima son los siguientes:
\begin{itemize}
\item Contaminación del suelo y del agua: La basura y los desechos generados por los turistas pueden contaminar el ecosistema.
\item Pérdida de biodiversidad: La presencia de turistas en las lomas puede alterar el hábitat de las especies animales y vegetales que viven en ellas, lo que puede llevar a la pérdida de biodiversidad.
\item Destrucción de ecosistemas frágiles: Las lomas costeras de Lima son ecosistemas frágiles que pueden ser fácilmente dañados por la presencia de turistas, lo que puede llevar a la degradación del ecosistema.
\end{itemize}

Las consecuencias de estos impactos pueden ser graves y de largo plazo:
\begin{itemize}
\item Afectación de la calidad del agua: La contaminación del agua puede llevar a la pérdida de fuentes de agua potable y afectar a las comunidades locales que dependen de estas fuentes.
\item Extinción de especies: La pérdida de biodiversidad puede resultar en la extinción de especies endémicas que son únicas de las lomas costeras de Lima.
\item Erosión del suelo: La destrucción de los ecosistemas frágiles puede llevar a la erosión del suelo, haciendo que las lomas sean menos estables y aumentando el riesgo de deslizamientos de tierra.
\end{itemize}



\section{III Objetivos de desarrollo sostenible}
\subsection{(ODS 12) Producción y Consumo Responsables}
La ODS 12 busca garantizar modalidades de consumo y producción sostenibles, promover la eficiencia en el uso de los recursos naturales y la energía, así como la generación de empleo y bienestar social.
\subsection{Metas específicas}
Una forma de poder aplicar este objetivo en el caso del turismo informal en las lomas costeras de Lima es a través de las siguientes recomendaciones:
\begin{itemize}
    \item Reducir la generación de residuos en las lomas costeras de Lima, a través de la implementación de programas de reciclaje y compostaje.
    \item Conservar la biodiversidad de las lomas costeras de Lima, a través de la protección de las especies animales y vegetales que viven en ellas.
  \end{itemize}
\subsection{(ODS 15) Vida de Ecosistemas Terrestres}
La ODS 15 busca proteger, restaurar y promover el uso sostenible de los ecosistemas terrestres, gestionar de forma sostenible los bosques, combatir la desertificación, detener e invertir la degradación de las tierras y detener la pérdida de biodiversidad.
\subsection{Metas específicas}
Un forma de poder aplicar este objetivo en el caso del turismo informal en las lomas costeras de Lima es a través de las siguientes acciones:
\begin{itemize}
    \item Proteger y conservar los ecosistemas terrestres de las lomas costeras de Lima, a través de la creación de áreas protegidas y la implementación de medidas de conservación.
    \item Restaurar los ecosistemas degradados de las lomas costeras de Lima, a través de la reforestación y la restauración de los suelos.
  \end{itemize}

\section{IV Propuesta de solución}
\subsection{Estrategias de mitigación del impacto ambiental.}
Para abordar este problema es importante antes revisar las medidas exitentes que, para este caso, se vienen cumpliendo en las lomas costeras de Lima. Por parte del gobierno provincial de Lima, se han establecido áreas de conservación regional para proteger las lomas costeras de Lima, como la Reserva Paisajística Nor Yauyos-Cochas y la Reserva Nacional Lomas de Lachay. Estas áreas protegidas tienen como objetivo conservar la biodiversidad y los ecosistemas frágiles de las lomas, así como promover el turismo sostenible y la educación ambiental. Sin embargo, a pesar de estos esfuerzos, el turismo informal sigue siendo una amenaza para las lomas costeras de Lima.

\subsubsection{Propuesta de solución innovadora: Sistema de certificación ambiental}

Una estrategia novedosa y viable para mitigar el impacto del turismo informal en las lomas costeras de Villa María del Triunfo es la implementación de un sistema de certificación ambiental para las empresas turísticas. Este sistema no es una práctica comúnmente observada en la región y puede ofrecer una solución integral a la problemática ambiental y al mismo tiempo estar acorde a los objetivos de desarrollo sostenible establecidos.

\begin{itemize}
  \item \textbf{Desarrollo del sistema de certificación:} Creación de un estándar de certificación ambiental específico para las lomas costeras, con criterios desarrollados en colaboración con expertos, ONGs y representantes del sector turístico.
  \item \textbf{Evaluación y cumplimiento:} Las empresas turísticas deberán someterse a evaluaciones para verificar el cumplimiento de los estándares ambientales, abarcando la gestión de residuos, la protección de la biodiversidad y el uso sostenible de los recursos.

\item \textbf{Incentivos económicos:} Incluir incentivos como reducción de impuestos y acceso a financiamiento preferencial para las empresas certificadas, facilitando la adopción de prácticas sostenibles.

\item \textbf{Monitoreo continuo:} Realización de auditorías ambientales periódicas por entidades independientes para asegurar el cumplimiento continuo de los estándares.

\item \textbf{Educación y concienciación:} Implementación de programas de educación y concienciación ambiental para operadores turísticos y visitantes, destacando la importancia de la conservación de las lomas.

\item \textbf{Participación comunitaria:} Involucrar a las comunidades locales en el desarrollo y ejecución del sistema de certificación, asegurando que sus necesidades y preocupaciones sean tomadas en cuenta.
\end{itemize}

\section{V Conclusiones y recomendaciones}
\subsection{Conclusiones}
El turismo informal en las lomas costeras de Lima puede tener un impacto negativo en el medio ambiente, debido a la falta de regulación y control de las actividades turísticas. Algunos de los impactos ambientales del turismo informal en las lomas costeras de Lima son la contaminación del suelo y del agua, la pérdida de biodiversidad y la destrucción de ecosistemas frágiles. Para mitigar el impacto del turismo informal en las lomas costeras de Lima, se propone la implementación de un sistema de certificación ambiental para las empresas turísticas que operan en las lomas.
\subsection{Recomendaciones}
\begin{enumerate}
  \item \textbf{Fortalecimiento de la vigilancia y el control:} Reforzar la vigilancia en las lomas de Villa María del Triunfo mediante patrullajes regulares y puntos de control para prevenir actividades ilegales y regular el acceso de visitantes.
  
  \item \textbf{Restauración de áreas degradadas:} Implementar programas de restauración ecológica en áreas degradadas, incluyendo reforestación con especies nativas y restauración de suelos, involucrando a comunidades locales y organizaciones ambientales.
  
  \item \textbf{Fomento de la investigación y la educación ambiental:} Promover la investigación científica sobre la biodiversidad de las lomas y establecer programas de educación ambiental en escuelas y comunidades para aumentar la conciencia y fomentar prácticas sostenibles.
  \end{enumerate}  
  
\newpage

\section*{VI Referencias}
\begin{enumerate}[label={[}\arabic*{]}]
  \item \textbf{MAPA NACIONAL DE ECOSISTEMAS DEL PERÚ.} Ministerio del Ambiente. Lima, 2011.\\
  \url{https://shorturl.at/DWAUE}

  \item \textbf{ECOSISTEMAS ÚNICOS: LAS LOMAS COSTERAS.}\\
  \url{https://www.minam.gob.pe/proyecolegios/Curso/curso-virtual/Modulos/modulo2/3Secundaria/Actividades-Aprendizaje/CTA_1/S11/anexo11/CTA_S11_Anexo_4.pdf}

  \item \textbf{ÁREAS DE CONSERVACIÓN REGIONAL LIMA.}\\
  \url{https://shorturl.at/mEUt9}

  \item \textbf{PROPUESTA DE ÁREA DE CONSERVACIÓN REGIONAL.}\\
  \url{https://pgrlm.gob.pe/wp-content/uploads/sites/30/2019/10/Sistema_de_Lomas.pdf}
  
\end{enumerate}
\end{document}
