\documentclass[11pt, a4paper]{article}
\usepackage[spanish]{babel}
\usepackage[utf8]{inputenc}
\usepackage{tikz}
\usepackage{graphicx}
\usepackage{amsmath}
\usepackage{amssymb}
\usepackage{geometry}
\title{\textbf{La recta en $R^2$}}
\author{Luis Huatay}

\begin{document}
\maketitle
\section{Retos - Semana 1-s1} Plano cartesiano, puntos medios y distancia entre dos puntos.
\subsection{Ejercicio 1.}

\textsl{\textbf{Demostrar que los puntos $A(0,1), B(3,5), C(7,2), D(4,-2)$ son los vértices de un cuadrado.}}
\\
\\
De acuerdo al gráfico:
\begin{figure}[h]
\centering
  \begin{tikzpicture}
    %ejex
    \draw[thick] (-0.5,0) -- (8,0) node[right]{$x$};
    %ejey
    \draw[thick] (0,-2.5) -- (0,5.5) node[above]{$y$};
    %cuadrícula
    \draw[help lines,gray,dashed,very thin] (-0.5,-2.5) grid (8,5.5);
    %puntos
    \draw [fill=black] (0,1) circle [radius=0.1] node[above left]{$A(0,1)$};
    \draw [fill=black] (3,5) circle [radius=0.1] node[above right]{$B(3,5)$};
    \draw [fill=black] (7,2) circle [radius=0.1] node[above right]{$C(7,2)$};
    \draw [fill=black] (4,-2) circle [radius=0.1] node[above right]{$D(4,-2)$};
    %líneas
    \draw[thick, red] (0,1) -- (3,5) -- (7,2) -- (4,-2) -- cycle;
  \end{tikzpicture}
  \caption{Representación gráfica}
  \label{fig:my_label}
\end{figure}
\\
\\

Para demostrar ello se consideralo siguiente:
\begin{itemize}
    \item Cada lado debe medir lo mismo.
    \item Los ángulos internos deben medir $90$\textdegree.
\end{itemize}
Para el primer aspecto, se calcula la distancia entre los puntos:
\begin{itemize}
    \item $AB = \sqrt{(3-0)^2 + (5-1)^2} = \sqrt{9+16} = \sqrt{25} = 5$
    \item $BC = \sqrt{(7-3)^2 + (2-5)^2} = \sqrt{16+9} = \sqrt{25} = 5$
    \item $CD = \sqrt{(4-7)^2 + (-2-2)^2} = \sqrt{9+16} = \sqrt{25} = 5$
    \item $DA = \sqrt{(0-4)^2 + (1+2)^2} = \sqrt{16+9} = \sqrt{25} = 5$
\end{itemize}
Para el segundo aspecto se considera la propiedad:
\begin{equation*}
  L_1 \perp L_2 \leftrightarrow m1 \cdot m2 = -1
\end{equation*}
Donde $m1$ y $m2$ son las pendientes de las rectas $L_1$ y $L_2$ respectivamente.\\
Para ello se calcula la pendiente de cada lado:
\begin{itemize}
    \item $m_{AB} * m_{BC} = \frac{5-1}{3-0} * \frac{2-5}{7-3} = \frac{-12}{12} = -1$
    \item $m_{BC} * m_{CD} = \frac{2-5}{7-3} * \frac{-2-2}{4-7} = \frac{12}{-12} = -1$
    \item $m_{CD} * m_{DA} = \frac{-2-2}{4-7} * \frac{-2-1}{4-0} = \frac{12}{-12} = -1$
    \item $m_{DA} * m_{AB} = \frac{-2-1}{4-0} * \frac{5-1}{3-0} = \frac{-12}{12} = -1$
\end{itemize}
\end{document}