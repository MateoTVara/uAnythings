\documentclass[11pt, a4paper]{article}
\usepackage[spanish]{babel}
\usepackage[utf8]{inputenc}
\usepackage{tikz}
\usepackage{titling}
\usepackage{graphicx}
\usepackage{amsmath}
\usepackage{amssymb}
\usepackage{geometry}
\usepackage{multicol}
\usepackage{cancel}
\title{\textbf{Ecuaciones lineales y sistemas de ecuaciones por métodos algebraicos}}
\author{Luis Huatay}

\begin{document}
\newgeometry{top=9cm}
\maketitle
\begin{center}
  Resolución de los retos de la semana 11, sesión 1 y 2.
\end{center}
\restoregeometry

\newpage
\newgeometry{
  top=3cm,
  bottom=3cm
}
\section{Sesión 1}
\subsection{Resolver la ecuación e indicar su conjunto solución:}
\begin{align*}
  2x - \frac{2-3x}{4} - \frac{5+x}{5} &= 2(x-3) - \frac{3}{5}\\
  \left(\frac{20}{20}\right) 2x - \left(\frac{5}{5}\right)\frac{2-3x}{4} - \left(\frac{4}{4}\right) \frac{5+x}{5} &= 2x - 6 - \frac{3}{5}\\
  \frac{40x - \left(10-15x\right) - \left(20+4x\right)}{20} &= 2x - 6 - \frac{3}{5}\\
  \frac{40x - 10 + 15x - 20 - 4x}{20} &= 2x - 6 - \frac{3}{5}\\
  40x - 10 + 15x - 20 - 4x &= 20\left(2x - 6 - \frac{3}{5}\right)\\
  \cancel{40x} - 30 + 11x &= \cancel{40x} - 120 - 12\\
  11x - \cancel{30 + 30} &= -132 +30\\
  11x &= -102\\
  x &= -\frac{102}{11} & \therefore \boxed{C.S. = \left\{-\frac{102}{11}\right\}}
\end{align*}
\\
\subsection{Resolver la ecuación e indicar su conjunto solución:}
\begin{align*}
  \frac{3x-2}{4} - \frac{5x-1}{3} &= \frac{2x-7}{6}\\
  \left(\frac{3}{3}\right)\frac{3x-2}{4} - \left(\frac{4}{4}\right)\frac{5x-1}{3} &= \frac{2x-7}{6}\\
  \frac{9x-6-20x+4}{12} &= \frac{2x-7}{6}\\
  -11x - 2 &= \cancel{12}\left(\frac{2x-7}{\cancel{6}}\right)\\
  -11x - 2 &= 2 \left(2x - 7\right)\\
  -11x - 2 &= 4x - 14\\
  -11x - 4x &= -14 + 2\\
  -15x &= -12\\
  x &= \frac{12}{15} & \therefore \boxed{C.S. = \left\{-\frac{12}{15}\right\}}
\end{align*}
\newpage
\subsection{Una compañía de telefonía celular cobra una cuota mensual de 10 soles por los primeros 15 Gb de consumo y 1.5 soles por cada Gb adicional. La cuenta de Jorge en un mes es de 70 soles. ¿Cuántos Gb ha consumido Jorge durante el mes?}
\begin{multicols}{2}
  \textbf{Considerese:}
  \begin{align*}
  cuota\ mensual &= 70 \\
  15\ Gb\ iniciales &= 10 \\
  Gb\ adicional &= 1.5 = \frac{3}{2}
  \end{align*}

  \columnbreak % Este comando divide las columnas

  \textbf{Luego:}
  \begin{align*}
    60 &= \frac{3}{2} x \\
    120 &= 3x \\
    x &= 40
  \end{align*}
\end{multicols}
\begin{align*}
  \textbf{Finalmente:}\\
  Cantidad\ de\ Gb\ adicionales &= 40\\
  Cantidad\ de\ Gb\ primer mes &= 15\\
  \therefore Cantidad\ de\ Gb\ total &= 15 + 40 = 55
\end{align*}
\subsection{Carmen está ahorrando para comprarse un departamento. Ella llega a heredar algún dinero de un familiar cercano, luego esto lo combina con 22 000 dólares que tenía ahorrado y duplica el total en una inversión afortunada. Ella termina con 135 000 dólares, que es justo lo que debe pagar para comprarse el departamento. ¿Cuánto heredó?}
\begin{multicols}{2}
  \textbf{Considerese:}
  \begin{align*}
  Herencia &= x \\
  Ahorro &= 22000\\
  Total &= 2y = 135000
  \end{align*}

  \columnbreak % Este comando divide las columnas

  \textbf{Luego:}
  \begin{align*}
    Ahorro + x &= 2y\\
  \end{align*}
\end{multicols}

\begin{align*}
  \textbf{Finalmente:}\\
  22000 + x &= 2y \\
  22000 + x &= 67500 \\
  x &= 45500\\
  & \therefore Se\ hereda\ inicialmente\ 45500
\end{align*}
\newpage
\newgeometry{
  top=1.5cm,
  bottom=1.5cm
}
\section{Sesión 2}
\subsection{Resolver el sistema de ecuaciones lineales mediante métodos algebraicos:}
\vspace{-1cm}
\begin{align*}
  \left.
    \begin{array}{rcl}
      x-y+2z &= &2\\
      3x+y+5z &= &8\\
      2x-y-2z &= &-7
    \end{array}
  \right\}
\end{align*}
\begin{multicols}{2}
  \textbf{Resolviendo en 1 y 2:}
  \begin{align*}
    \left.
      \begin{array}{rcl}
        x-y+2z &= &2\\
        3x+y+5z &= &8
      \end{array}
    \right\} = 4x + 7z &= 10
  \end{align*}
  \textbf{Resolviendo en 2 y 3:}
  \begin{align*}
    \left.
      \begin{array}{rcl}
        3x+y+5z &= &8\\
        2x-y-2z &= &-7
      \end{array}
    \right\} = 5x + 3z &= 1
  \end{align*}
  \textbf{Calculando x:}
  \begin{align*}
    \left.
      \begin{array}{rcl}
        4x + 7z &= 10\\
        5x + 3z &= 1
      \end{array}
    \right\}\\
    x = \frac{30 - 7}{12 - 35} &= \frac{23}{-23}\\
    &\boxed{x = -1}
  \end{align*}

  \columnseprule=1pt 
  \columnbreak

  \textbf{Calculando z:}
  \begin{align*}
    5x + 3z &= 1\\
        5\left(-1\right) + 3z &= 1\\
        -5 + 3z &= 1\\
        3z &= 6\\
        &\boxed{z = 2}
  \end{align*}
  \textbf{Luego reemplanzando x, z en 2:}
  \begin{align*}
    3x+y+5z &= 8\\
    3\left(-1\right) + y + 5\left(2\right) &= 8\\
    -3 + y + 10 &= 8\\
    &\boxed{y = 1}
  \end{align*}
\end{multicols}
\textbf{Finalmente:}
\begin{align*}
  C.S. \left(x,y,z\right) = \left\{(-1, 1, 2)\right\}
\end{align*}
\subsection{Resolver el sistema de ecuaciones lineales mediante métodos algebraicos:}
\vspace{-1cm}
\begin{align*}
  \left.
    \begin{array}{rcl}
      2x+y+z &= &8\\
      3x-2y-z &= &1\\
      4x-7y+3z &= &10
    \end{array}
  \right\}
\end{align*}
\begin{multicols}{2}
  \textbf{Resolviendo en 1 y 2:}
  \begin{align*}
    \left.
      \begin{array}{rcl}
        2x+y+\cancel{z} &= &8\\
      3x-2y-\cancel{z} &= &1
      \end{array}
    \right\} = 5x - y = 9\\
  \end{align*}

  \textbf{Resolviendo en 2 y 3:}
  \begin{align*}
    \left.
      \begin{array}{rcl}
        \left(3\right)3x-2y-z &= &1\left(3\right)\\
        4x-7y+3z &= &10
      \end{array}
    \right\}\\
    \left.
      \begin{array}{rcl}
        9x-6y-\cancel{3z} &= &3\\
        4x-7y+\cancel{3z} &= &10
      \end{array}
    \right\}\\
    \left(\frac{-1}{13}\right)13x-13y = 13\left(\frac{-1}{13}\right)\\
    -x + y = -1
  \end{align*}
  \columnseprule=1pt 
  \textbf{Calculando x:}
  \begin{align*}
    \left.
      \begin{array}{rcl}
        5x - \cancel{y} &= &9\\
        -x + \cancel{y} &= &-1
      \end{array}
    \right\}\\
    4x = 8\\
    &\boxed{x = 2}
  \end{align*}
  \textbf{Calculando y:}
  \begin{align*}
    -x + y &= -1\\
    -\left(2\right) + y &= -1\\
    -2 + y &= -1\\
    &\boxed{y = 1}
  \end{align*}

  \textbf{Luego reemplanzando x, y en 2:}
  \begin{align*}
    3x-2y-z &= 1\\
    3\left(2\right) -2\left(-1\right)-z &=1\\
    6 - 2 - z &= 1\\
    4 - z &= 1\\
    &\boxed{z = 3}
  \end{align*}
\end{multicols}
\textbf{Finalmente:}
\begin{align*}
  C.S. \left(x,y,z\right) = \left\{(2,1,3)\right\}
\end{align*}
\newpage
\subsection{Un puesto de frutas vende dos variedades de frutas: estándar y de lujo. Una caja de fresas estándar se vende a \$7 y una de lujo se vende a \$10. En un día, el puesto vende 135 cajas de fresa en un total de \$1101. ¿Cuántas cajas de cada tipo se vendieron?}
\begin{multicols}{2}
  \textbf{Considerese:}
  \begin{align*}
  Precio\ de\ caja\ estandar &= 7\\
  Precio\ de\ caja\ de\ lujo &= 10\\
  Total\ de\ cajas &= 135\\
  Total\ de\ dinero &= 1101
  \end{align*}

  \textbf{Luego:}
  \begin{align*}
    Cajas\ de\ fresas\ estandar &= x \\
    Cajas\ de\ fresas\ de\ lujo &= y\\
    \left.
      \begin{array}{rcl}
        7x + 10y &= &1101\\
        x + y &= &135
        \end{array}
    \right\}
  \end{align*}\\

  \columnseprule=1pt 
  \columnbreak

  \textbf{Calculando x:}
  \begin{align*}
    \left.
      \begin{array}{rcl}
        7x + 10y &= &1101\\
        \left(-10\right)x + y &= &135\left(-10\right)
        \end{array}
    \right\}\\
    \left.
      \begin{array}{rcl}
        7x + \cancel{10y} &= &1101\\
        -10x -\cancel{10y} &= &-1350
        \end{array}
    \right\}\\
    -3x = -249\\
    \boxed{x = 83}
  \end{align*}


  \textbf{Calculando y:}
  \begin{align*}
    x + y &= 135\\
    83 + y &= 135\\
    &\boxed{y = 52}
  \end{align*}
\end{multicols}
\textbf{Finalmente:}
Se vendieron 83 cajas de fresas estándar y 52 cajas de fresas de lujo.
\subsection{Un agricultor tiene 1200 acres de tierras en las que produce maíz, trigo y frijol de soya. Cuesta \$45 por acre producir maiz, \$60 producir trigo y \$50 producir frijol de soya. Debido a la demanda del mercado, el agricultor producirá el doble de acres de trigo que de maíz. Ha asignado \$63750 para el costo de producir sus cosechas. ¿Cuántos acres de cada cultivo debe plantar?.}
\begin{multicols}{2}
  \textbf{Considerese:}
  \begin{align*}
  Acres de Maiz &= x\\
  Acres de Trigo &= y\\
  Acres de Frijol de Soya &= z
  \end{align*}
  \textbf{Expresión dada por el área total:}
  \begin{align*}
    x + y + z &= 1200
  \end{align*}
  \textbf{Expresión dada por la demanda:}
  \begin{align*}
    x + 2x + z &= 1200\\
    3x + z &= 1200
  \end{align*}
  \textbf{Expresión dada por la inversión:}
  \begin{align*}
    45x + 60y + 50z &= 63750
  \end{align*}  
  \textbf{Dado el sistema:}
  \begin{align*}
    \left.
      \begin{array}{rcl}
        45x + 60y + 50z &= &63750\\
        \left(-60\right)x + y + z &= &1200\left(-60\right)
      \end{array}
    \right\}\\
    \left.
      \begin{array}{rcl}
        45x + \cancel{60y} + 50z &= &63750\\
        -60x - \cancel{60y} - 60z &= &-72000
      \end{array}
    \right\}\\
    -15x - 10z = -8250
  \end{align*}\\
  \columnseprule=1pt
  \columnbreak
  \\
  \textbf{Luego reemplazando en área:}
  \begin{align*}
    \left.
      \begin{array}{rcl}
        -15x - 10z &= &-8250\\
        \left(10\right)3x + z &= &1200\left(10\right)
      \end{array}
    \right\}\\
    \left.
      \begin{array}{rcl}
        -15x - \cancel{10z} &= &-8250\\
        30x + \cancel{10z} &= &12000
      \end{array}
    \right\}\\
    15x = 3750\\
    \boxed{x = 250}
  \end{align*}
  \textbf{Hallando z:}
  \begin{align*}
    3x + z &= 1200\\
    3\left(250\right) + z &= 1200\\
    750 + z &= 1200
    &\boxed{z = 450}
  \end{align*}
  \textbf{Hallando y:}
  \begin{align*}
    x + y + z &= 1200\\
    250 + y + 450 &= 1200\\
    y &= 1200-700\\&\boxed{y = 500}
  \end{align*}
  \textbf{El granjero ocupa 250 acres de maíz, 500 acres de trigo y 450 acres de frijol de soya.}
\end{multicols}
\end{document}